\documentclass[12pt]{article}
\usepackage[utf8]{inputenc}
\usepackage[spanish]{babel}
\usepackage{amsmath,amssymb}
\usepackage{booktabs}
\usepackage{xcolor}
\usepackage{graphicx}
\usepackage{geometry}
\geometry{margin=2.5cm}
\usepackage{fancyhdr}
\setlength{\headheight}{14.5pt}
\pagestyle{fancy}
\fancyhf{}
\rhead{Investigación de Operaciones}
\lhead{Método Simplex}

\definecolor{basecolor}{RGB}{0,128,0}
\definecolor{entracolor}{RGB}{255,0,0}
\definecolor{entracolor}{RGB}{0,0,255}
\definecolor{empatecolor}{RGB}{0,0,200}
\definecolor{pivotecolor}{RGB}{200,0,200}

\title{Resultados del Método Simplex\\
\large Problema: \textbf{Degenerado}}
\author{
Emily Sánchez\\
Viviana Vargas\\
\\
Curso: Investigación de Operaciones\\
Semestre II: 2025
}
\date{\today}

\begin{document}

\maketitle
\thispagestyle{empty}

\newpage
\tableofcontents
\newpage
\section{El Algoritmo Simplex}

\subsection{Historia}
El m\'etodo Simplex fue desarrollado por George Dantzig en 1947 mientras trabajaba para la Fuerza A\'erea de los Estados Unidos. \\

Es uno de los algoritmos m\'as importantes en la historia de la optimizaci\'on matem\'atica y ha sido fundamental en el desarrollo de la programaci\'on lineal. Usa operaciones sobre matrices hasta encontrar la soluci\'on \'optima o determinar que el problema no tiene soluci\'on. Parte de un v\'ertice de la regi\'on factible y "salta" a v\'ertices adyacentes que mejoren lo encontrado hasta encontrar la condici\'on de salida.

\subsection{Complejidad}
En el peor de los casos el S\'implex podr\'ia probar tantas bases como el m\'etodo exhaustivo. (Solo en ejemplos montados para que falle.) Normalmente va a encontrar la soluci\'on \'optima en $3n$ intentos.

\subsection{Propiedades Fundamentales}
\begin{itemize}
\item \textbf{Convergencia:} El algoritmo converge a la soluci\'on \'optima en un n\'umero finito de pasos (en la mayor\'ia de los casos pr\'acticos)
\item \textbf{Complejidad:} En el peor caso tiene complejidad exponencial, pero en la pr\'actica es muy eficiente
\item \textbf{Optimalidad:} Garantiza encontrar la soluci\'on \'optima global para problemas convexos
\item \textbf{Factibilidad:} Mantiene la factibilidad en cada iteraci\'on
\end{itemize}

\subsection{Aplicaciones}
S\'implex se usa en aplicaciones profesionales como la optimizaci\'on de recursos empresariales, la log\'istica y la gesti\'on de proyectos, donde se usa para encontrar la soluci\'on m\'as eficiente a problemas complejos con m\'ultiples variables y restricciones.

\subsection{Descripci\'on del M\'etodo}
El m\'etodo Simplex opera movi\'endose entre v\'ertices adyacentes del poliedro factible, mejorando el valor de la funci\'on objetivo en cada paso hasta alcanzar el \'optimo.

\section{Pasos del M\'etodo Simplex}
\begin{enumerate}
\item \textbf{Formulaci\'on del problema:}
\begin{itemize}
\item Define la funci\'on objetivo a maximizar o minimizar y las restricciones lineales asociadas.
\item Aseg\'urate de que todas las variables de decisi\'on sean no negativas.
\end{itemize}

\item \textbf{Conversi\'on a forma est\'andar:}
\begin{itemize}
\item Transforma todas las restricciones de desigualdad en igualdades introduciendo variables de holgura (sumar una variable para restricciones $\leq$) o variables de exceso (restar una variable para restricciones $\geq$).
\item Aseg\'urate de que la funci\'on objetivo tambi\'en se iguale a cero.
\end{itemize}

\item \textbf{Creaci\'on de la tabla inicial:}
\begin{itemize}
\item Organiza el problema en una tabla llamada Tabla Simplex.
\item La tabla inicial se construye con las variables del problema y una matriz identidad formada por las variables de holgura/exceso.
\end{itemize}

\item \textbf{Iteraci\'on:}
\begin{itemize}
\item \textbf{Selecci\'on de la variable entrante:} Elige la columna con el valor m\'as negativo en la fila de la funci\'on objetivo (para maximizar) o el m\'as positivo (para minimizar). Esta es la columna que entra en la base.
\item \textbf{Selecci\'on de la variable saliente:} Calcula la raz\'on entre el lado derecho de cada restricci\'on y el valor correspondiente en la columna de la variable entrante. La fila con el cociente m\'as peque\~no es la que sale de la base.
\item \textbf{Pivoteo:} Realiza operaciones de escalerizaci\'on para hacer que el elemento en la intersecci\'on de la fila entrante y la columna saliente sea $1$ y todos los dem\'as elementos en esa columna sean $0$. Esto actualiza la tabla y la soluci\'on.
\end{itemize}

\item \textbf{Comprobaci\'on de la soluci\'on \'optima:}
\begin{itemize}
\item Verifica si la fila de la funci\'on objetivo en la tabla ya no contiene coeficientes negativos (para maximizar) o positivos (para minimizar).
\item Si es as\'i, la soluci\'on es \'optima y se puede leer directamente de la tabla.
\item Si no, repite el proceso de iteraci\'on.
\end{itemize}
\end{enumerate}
\section{Problema Original}

\subsection{Formulación Matemática}
\textbf{Problema de Maximización}

\[ \text{Maximizar } Z = 2x_{1} + 1x_{2} \]

\textbf{Sujeto a:}
\begin{align*}
3x_{1} + 1x_{2} &\leq 6 \\
1x_{1} -1x_{2} &\leq 2 \\
0x_{1} + 1x_{2} &\leq 3
\end{align*}

\textbf{Con:}
\[ x_{1} \geq 0, \quad x_{2} \geq 0 \]

\section{Tabla Inicial del Método Simplex}

La tabla inicial del método Simplex se construye agregando variables de holgura para convertir las desigualdades en igualdades.

\begin{center}
\small
\begin{tabular}{|c|c|c|c|c|c|c|c|}
\hline
\textbf{Variable} & \textbf{Z} & \textbf{$x_{1}$} & \textbf{$x_{2}$} & \textbf{$S_1$} & \textbf{$S_2$} & \textbf{$S_3$} & \textbf{b} \\
\hline
\textbf{Z} & 1 & -2 & -1 & 0 & 0 & 0 & 0 \\
\hline
\textbf{$S_1$} & 0 & 3 & 1 & 1 & 0 & 0 & 6 \\
\hline
\textbf{$S_2$} & 0 & 1 & -1 & 0 & 1 & 0 & 2 \\
\hline
\textbf{$S_3$} & 0 & 0 & 1 & 0 & 0 & 1 & 3 \\
\hline
\end{tabular}
\end{center}

\item \textbf{$S_1$:} Coeficientes de las variables de holgura
\item \textbf{$S_2$:} Coeficientes de las variables de holgura
\item \textbf{$S_3$:} Coeficientes de las variables de holgura
\item \textbf{b:} T\'erminos independientes (lado derecho)
\item \textbf{Base inicial:} Variables de holgura $S_1, S_2, \ldots, S_3$
\end{itemize}

\section{Proceso Iterativo del Método Simplex}

A continuación se detalla el proceso iterativo del algoritmo Simplex, mostrando cada tabla y las operaciones de pivoteo realizadas.

\subsection{Tabla Inicial}
\textbf{Estado inicial:} Variables de holgura en la base.\par\smallskip

\begin{center}
\small
\begin{tabular}{|c|c|c|c|c|c|c|c|}
\hline
\textbf{Variable} & \textbf{Z} & \textbf{$x_{1}$} & \textbf{$x_{2}$} & \textbf{$S_1$} & \textbf{$S_2$} & \textbf{$S_3$} & \textbf{b} \\
\hline
\textbf{Z} & 1 & \textcolor{red}{-2} & \textcolor{red}{-1} & 0 & 0 & 0 & 0 \\
\hline
\textbf{\textcolor{green}{$S_1$}} & 0 & 3 & 1 & 1 & 0 & 0 & 6 \\
\hline
\textbf{\textcolor{green}{$S_2$}} & 0 & 1 & -1 & 0 & 1 & 0 & 2 \\
\hline
\textbf{\textcolor{green}{$S_3$}} & 0 & 0 & 1 & 0 & 0 & 1 & 3 \\
\hline
\end{tabular}
\end{center}

\textbf{Variables en la base:} \textcolor{green}{$S_1$}, \textcolor{green}{$S_2$}, \textcolor{green}{$S_3$}\\

\subsection{Iteración 1}
\textbf{Operación de pivoteo:}
\begin{itemize}
\item \textbf{Variable que entra:} $x_{1}$
\item \textbf{Variable que sale:} $S_1$
\item \textbf{Elemento pivote:} 3 (fila 1, columna 1)
\item \textcolor{blue}{\textbf{Empate detectado:}} 2 filas con ratio mínimo\\
\quad \textcolor{blue}{Filas empatadas:} 1, 2\\
\quad \textcolor{blue}{Selección:} Fila 1 (aleatoria)
\end{itemize}

\begin{center}
\small
\begin{tabular}{|c|c|c|c|c|c|c|c|}
\hline
\textbf{Variable} & \textbf{Z} & \textbf{$x_{1}$} & \textbf{$x_{2}$} & \textbf{$S_1$} & \textbf{$S_2$} & \textbf{$S_3$} & \textbf{b} \\
\hline
\textbf{Z} & 1 & 0 & \textcolor{red}{-0.33} & 0.67 & 0 & 0 & 4 \\
\hline
\textcolor{blue}{\textbf{$x_{1}$}} & 0 & \mathbf{[1]} & \textcolor{blue}{0.33} & \textcolor{blue}{0.33} & \textcolor{blue}{0} & \textcolor{blue}{0} & \textcolor{blue}{2} \\
\hline
\textcolor{blue}{\textbf{$S_2$}} & 0 & \textcolor{blue}{0} & \textcolor{blue}{-1.33} & \textcolor{blue}{-0.33} & \textcolor{blue}{1} & \textcolor{blue}{0} & \textcolor{blue}{0} \\
\hline
\textbf{\textcolor{green}{$S_3$}} & 0 & 0 & 1 & 0 & 0 & 1 & 3 \\
\hline
\end{tabular}
\end{center}

\noindent\textcolor{blue}{$\blacksquare$}\hspace{0.5em}\textbf{Filas en azul:} empates en la selección del pivote\\

\textbf{Variables en la base:} \textcolor{blue}{$x_{1}$}, \textcolor{blue}{$S_2$}, \textcolor{green}{$S_3$}\\

\subsection{Iteración 2}
\textbf{Operación de pivoteo:}
\begin{itemize}
\item \textbf{Variable que entra:} $x_{2}$
\item \textbf{Variable que sale:} $S_3$
\item \textbf{Elemento pivote:} 1 (fila 3, columna 2)
\end{itemize}

\begin{center}
\small
\begin{tabular}{|c|c|c|c|c|c|c|c|}
\hline
\textbf{Variable} & \textbf{Z} & \textbf{$x_{1}$} & \textbf{$x_{2}$} & \textbf{$S_1$} & \textbf{$S_2$} & \textbf{$S_3$} & \textbf{b} \\
\hline
\textbf{Z} & 1 & 0 & 0 & 0.67 & 0 & 0.33 & 5 \\
\hline
\textbf{\textcolor{green}{$x_{1}$}} & 0 & 1 & 0 & 0.33 & 0 & -0.33 & 1 \\
\hline
\textbf{\textcolor{green}{$S_2$}} & 0 & 0 & 0 & -0.33 & 1 & 1.33 & 4 \\
\hline
\textbf{\textcolor{green}{$x_{2}$}} & 0 & 0 & \mathbf{[1]} & 0 & 0 & 1 & 3 \\
\hline
\end{tabular}
\end{center}

\textbf{Variables en la base:} \textcolor{green}{$x_{1}$}, \textcolor{green}{$S_2$}, \textcolor{green}{$x_{2}$}\\

\section{Tabla Final}

La siguiente tabla representa la solución óptima del problema:

\begin{center}
\small
\begin{tabular}{|c|c|c|c|c|c|c|}
\hline
 & \textbf{$x_{1}$} & \textbf{$x_{2}$} & \textbf{$s_1$} & \textbf{$s_2$} & \textbf{$s_3$} & \textbf{L.D.} \\
\hline
\textbf{Z} & 0 & 0 & 0.67 & 0 & 0.33 & 5 \\
\hline
\textbf{$x_{1}$} & 1 & 0 & 0.33 & 0 & -0.33 & 1 \\
\hline
\textbf{$s_2$} & 0 & 0 & -0.33 & 1 & 1.33 & 4 \\
\hline
\textbf{$x_{2}$} & 0 & 1 & 0 & 0 & 1 & 3 \\
\hline
\end{tabular}
\end{center}

\section{Solución Óptima}

\textbf{Valor óptimo de la función objetivo: } $Z = 5$

\textbf{Valores de las variables:}
\begin{align*}
x_{1} &= 1 \\
x_{2} &= 3 \\
S_2 &= 4 \\
S_1 &= 0 \\
S_3 &= 0
\end{align*}

\textbf{Tipo:} Problema Degenerado

\subsection{Explicación del Problema Degenerado}

Un problema de programación lineal se considera \textbf{degenerado} cuando al menos una variable básica toma el valor cero en la solución óptima.

\textbf{Características de la degeneración:}
\begin{itemize}
\item Al menos una variable básica tiene valor cero
\item Puede ocurrir cuando hay restricciones redundantes
\item Puede llevar a ciclado en el algoritmo Simplex (aunque es raro en la práctica)
\item La solución óptima puede no ser única
\item En problemas prácticos, la degeneración es común pero generalmente no afecta la calidad de la solución
\end{itemize}

\textbf{Causas comunes:}
\begin{itemize}
\item Restricciones redundantes en el modelo
\item Múltiples restricciones que se intersectan en el mismo punto
\item Problemas con estructura especial que genera empates en la selección de variables
\end{itemize}

\textbf{Manejo en el algoritmo Simplex:}
\begin{itemize}
\item Se utiliza una tolerancia numérica ($\\epsilon = 10^{-10}$) para detectar valores cero
\item Cuando hay empates en la regla del ratio mínimo, se elige arbitrariamente
\item La elección arbitraria evita el ciclado en la mayoría de los casos prácticos
\item En este problema se realizaron 2 iteraciones sin ciclado
\end{itemize}

\textbf{Implicaciones prácticas:}
\begin{itemize}
\item La solución encontrada es válida y óptima
\item Puede existir más de una solución óptima (soluciones alternativas)
\item En aplicaciones prácticas, la degeneración generalmente no es problemática
\item Si es necesario, se pueden usar técnicas anti-ciclado (regla de Bland)
\end{itemize}

\textbf{Variables básicas con valor cero (degeneradas):}
\begin{itemize}
\end{itemize}

\textbf{Iteraciones realizadas:} 2

\textbf{Observaciones:} Se detectó degeneración (1 variables básicas = 0)

\end{document}
