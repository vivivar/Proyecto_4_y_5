\documentclass[12pt]{article}
\usepackage[utf8]{inputenc}
\usepackage[spanish]{babel}
\usepackage{amsmath,amssymb}
\usepackage{booktabs}
\usepackage{xcolor}
\usepackage{graphicx}
\usepackage{geometry}
\geometry{margin=2.5cm}
\usepackage{fancyhdr}
\setlength{\headheight}{14.5pt}
\pagestyle{fancy}
\fancyhf{}
\rhead{Investigación de Operaciones}
\lhead{Método Simplex}

\definecolor{basecolor}{RGB}{0,128,0}
\definecolor{entracolor}{RGB}{255,0,0}

\title{Resultados del Método Simplex\\
\large Problema: \textbf{Multiples2}}
\author{
Emily Sánchez\\
Viviana Vargas\\
\\
Curso: Investigación de Operaciones\\
Semestre II: 2025
}
\date{\today}

\begin{document}

\maketitle
\thispagestyle{empty}

\newpage
\tableofcontents
\newpage
\section{El Algoritmo Simplex}

\subsection{Historia}
El método Simplex fue desarrollado por George Dantzig en 1947 mientras trabajaba para la Fuerza Aérea de los Estados Unidos. \\

Es uno de los algoritmos más importantes en la historia de la optimización matemática y ha sido fundamental en el desarrollo de la programación lineal.

\subsection{Propiedades Fundamentales}
\begin{itemize}
\item \textbf{Convergencia:} El algoritmo converge a la solución óptima en un número finito de pasos (en la mayoría de los casos prácticos)
\item \textbf{Complejidad:} En el peor caso tiene complejidad exponencial, pero en la práctica es muy eficiente
\item \textbf{Optimalidad:} Garantiza encontrar la solución óptima global para problemas convexos
\item \textbf{Factibilidad:} Mantiene la factibilidad en cada iteración
\end{itemize}

\subsection{Descripción del Método}
El método Simplex opera moviéndose entre vértices adyacentes del poliedro factible, mejorando el valor de la función objetivo en cada paso hasta alcanzar el óptimo.

\section{Problema Original}

\subsection{Formulación Matemática}
\textbf{Problema de Maximización}

\[ \text{Maximizar } Z = 4x_{1} + 14x_{2} \]

\textbf{Sujeto a:}
\begin{align*}
2x_{1} + 7x_{2} &\leq 21 \\
7x_{1} + 2x_{2} &\leq 21
\end{align*}

\textbf{Con:}
\[ x_{1} \geq 0, \quad x_{2} \geq 0 \]

\section{Tabla Inicial del Método Simplex}

La tabla inicial del método Simplex se construye agregando variables de holgura para convertir las desigualdades en igualdades.

\begin{center}
\small
\begin{tabular}{|c|c|c|c|c|c|c|}
\hline
\textbf{Variable} & \textbf{Z} & \textbf{$x_{1}$} & \textbf{$x_{2}$} & \textbf{$S_1$} & \textbf{$S_2$} & \textbf{b} \\
\hline
\textbf{Z} & 1 & -4 & -14 & 0 & 0 & 0 \\
\hline
\textbf{$S_1$} & 0 & 2 & 7 & 1 & 0 & 21 \\
\hline
\textbf{$S_2$} & 0 & 7 & 2 & 0 & 1 & 21 \\
\hline
\end{tabular}
\end{center}

\item \textbf{x_{1}:} Coeficientes de las variables de decisi\'on
\item \textbf{x_{2}:} Coeficientes de las variables de decisi\'on
\item \textbf{$S_1$:} Coeficientes de las variables de holgura
\item \textbf{$S_2$:} Coeficientes de las variables de holgura
\item \textbf{b:} T\'erminos independientes (lado derecho)
\item \textbf{Base inicial:} Variables de holgura $S_1, S_2, \ldots, S_2$
\end{itemize}

\section{Proceso Iterativo del Método Simplex}

A continuación se detalla el proceso iterativo del algoritmo Simplex, mostrando cada tabla y las operaciones de pivoteo realizadas.

\subsection{Tabla Inicial}
\textbf{Estado inicial:} Variables de holgura en la base.\par\smallskip

\begin{center}
\small
\begin{tabular}{|c|c|c|c|c|c|c|}
\hline
\textbf{Variable} & \textbf{Z} & \textbf{$x_{1}$} & \textbf{$x_{2}$} & \textbf{$S_1$} & \textbf{$S_2$} & \textbf{b} \\
\hline
\textbf{Z} & 1 & \textcolor{red}{-4} & \textcolor{red}{-14} & 0 & 0 & 0 \\
\hline
\textbf{\textcolor{green}{$S_1$}} & 0 & 2 & 7 & 1 & 0 & 21 \\
\hline
\textbf{\textcolor{green}{$S_2$}} & 0 & 7 & 2 & 0 & 1 & 21 \\
\hline
\end{tabular}
\end{center}

\textbf{Variables en la base:} \textcolor{green}{$S_1$}, \textcolor{green}{$S_2$}\\

\subsection{Iteración 1}
\textbf{Operación de pivoteo:}
\begin{itemize}
\item \textbf{Variable que entra:} $x_{2}$
\item \textbf{Variable que sale:} $S_1$
\item \textbf{Elemento pivote:} 7 (fila 1, columna 2)
\end{itemize}

\begin{center}
\small
\begin{tabular}{|c|c|c|c|c|c|c|}
\hline
\textbf{Variable} & \textbf{Z} & \textbf{$x_{1}$} & \textbf{$x_{2}$} & \textbf{$S_1$} & \textbf{$S_2$} & \textbf{b} \\
\hline
\textbf{Z} & 1 & 0 & 0 & 2 & 0 & 42 \\
\hline
\textbf{\textcolor{green}{$x_{2}$}} & 0 & 0.29 & \mathbf{[1]} & 0.14 & 0 & 3 \\
\hline
\textbf{\textcolor{green}{$S_2$}} & 0 & 6.43 & 0 & -0.29 & 1 & 15 \\
\hline
\end{tabular}
\end{center}

\textbf{Variables en la base:} \textcolor{green}{$x_{2}$}, \textcolor{green}{$S_2$}\\

\subsection{Iteración 2}
\begin{center}
\small
\begin{tabular}{|c|c|c|c|c|c|c|}
\hline
\textbf{Variable} & \textbf{Z} & \textbf{$x_{1}$} & \textbf{$x_{2}$} & \textbf{$S_1$} & \textbf{$S_2$} & \textbf{b} \\
\hline
\textbf{Z} & 1 & 0 & 0 & 2 & 0 & 42 \\
\hline
\textbf{\textcolor{green}{$x_{2}$}} & 0 & 0 & 1 & 0.16 & -0.04 & 2.33 \\
\hline
\textbf{\textcolor{green}{$x_{1}$}} & 0 & 1 & 0 & -0.04 & 0.16 & 2.33 \\
\hline
\end{tabular}
\end{center}

\textbf{Variables en la base:} \textcolor{green}{$x_{2}$}, \textcolor{green}{$x_{1}$}\\

\section{Tabla Final}

\textbf{Soluciones Múltiples:} Se encontraron dos tablas finales que representan diferentes soluciones óptimas.

\subsection{Primera Solución Óptima}
\begin{center}
\small
\begin{tabular}{|c|c|c|c|c|c|}
\hline
 & \textbf{$x_{1}$} & \textbf{$x_{2}$} & \textbf{$s_1$} & \textbf{$s_2$} & \textbf{L.D.} \\
\hline
\textbf{Z} & 0 & 0 & 2 & 0 & 42 \\
\hline
\textbf{$x_{2}$} & 0.29 & 1 & 0.14 & 0 & 3 \\
\hline
\textbf{$s_2$} & 6.43 & 0 & -0.29 & 1 & 15 \\
\hline
\end{tabular}
\end{center}

\subsection{Segunda Solución Óptima}
\begin{center}
\small
\begin{tabular}{|c|c|c|c|c|c|}
\hline
 & \textbf{X1} & \textbf{X2} & \textbf{$s_1$} & \textbf{$s_2$} & \textbf{L.D.} \\
\hline
\textbf{Z} & 0 & 0 & 2 & 0 & 42 \\
\hline
\textbf{$X2$} & 0 & 1 & 0.16 & -0.04 & 2.33 \\
\hline
\textbf{$X1$} & 1 & 0 & -0.04 & 0.16 & 2.33 \\
\hline
\end{tabular}
\end{center}

\section{Solución Óptima}

\textbf{Valor óptimo de la función objetivo: } $Z = 42$

\textbf{Valores de las variables:}
\begin{align*}
x_{1} &= 0 \\
x_{2} &= 3 \\
S_2 &= 15 \\
S_1 &= 0
\end{align*}

\textbf{Tipo:} Soluciones Múltiples

\subsection{Explicación de Soluciones Múltiples}

Cuando un problema de programación lineal tiene soluciones múltiples, significa que existe más de una combinación de valores para las variables de decisión que produce el mismo valor óptimo de la función objetivo.

\textbf{Condición para soluciones múltiples:}
\begin{itemize}
\item Al menos una variable no básica tiene coeficiente cero en la fila Z de la tabla óptima
\item Esto indica que podemos introducir esa variable en la base sin cambiar el valor de Z
\item El conjunto de soluciones óptimas forma un segmento de recta (en 2D) o un hiperplano (en nD)
\end{itemize}

\textbf{Ecuación para generar todas las soluciones óptimas:}
\[
X = \lambda X_1 + (1-\lambda) X_2, \quad 0 \leq \lambda \leq 1
\]
Donde $X_1$ y $X_2$ son las dos soluciones básicas encontradas y $\lambda$ es un parámetro entre 0 y 1.

\subsection{Soluciones Adicionales}

A continuación se presentan 3 soluciones adicionales obtenidas como combinaciones convexas de las dos soluciones básicas:

\subsubsection{Solución con $\lambda = 0,25$}
\begin{align*}
x_{1} &= 1.75 \\
x_{2} &= 2.50
\end{align*}

\textbf{Verificación: } $Z = 42$ (mismo valor óptimo)\par\smallskip

\subsubsection{Solución con $\lambda = 0,50$}
\begin{align*}
x_{1} &= 1.17 \\
x_{2} &= 2.67
\end{align*}

\textbf{Verificación: } $Z = 42$ (mismo valor óptimo)\par\smallskip

\subsubsection{Solución con $\lambda = 0,75$}
\begin{align*}
x_{1} &= 0.58 \\
x_{2} &= 2.83
\end{align*}

\textbf{Verificación: } $Z = 42$ (mismo valor óptimo)\par\smallskip

\textbf{Iteraciones realizadas:} 1

\textbf{Observaciones:} Se detectaron soluciones múltiples

\end{document}
