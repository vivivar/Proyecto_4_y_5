\documentclass[12pt]{article}
\usepackage[utf8]{inputenc}
\usepackage[spanish]{babel}
\usepackage{amsmath,amssymb}
\usepackage{booktabs}
\usepackage{xcolor}
\usepackage{graphicx}
\usepackage{geometry}
\geometry{margin=2.5cm}
\usepackage{fancyhdr}
\setlength{\headheight}{14.5pt}
\pagestyle{fancy}
\fancyhf{}
\rhead{Investigación de Operaciones}
\lhead{Método Simplex}

\definecolor{basecolor}{RGB}{0,128,0}
\definecolor{entracolor}{RGB}{255,0,0}
\definecolor{entracolor}{RGB}{0,0,255}
\definecolor{empatecolor}{RGB}{0,0,200}
\definecolor{pivotecolor}{RGB}{200,0,200}

\title{Resultados del Método Simplex\\
\large Problema: \textbf{Problema7}}
\author{
Emily Sánchez\\
Viviana Vargas\\
\\
Curso: Investigación de Operaciones\\
Semestre II: 2025
}
\date{\today}

\begin{document}

\maketitle
\thispagestyle{empty}

\newpage
\tableofcontents
\newpage
\section{El Algoritmo Simplex}

\subsection{Historia}
El m\'etodo Simplex fue desarrollado por George Dantzig en 1947 mientras trabajaba para la Fuerza A\'erea de los Estados Unidos. \\

Es uno de los algoritmos m\'as importantes en la historia de la optimizaci\'on matem\'atica y ha sido fundamental en el desarrollo de la programaci\'on lineal. Usa operaciones sobre matrices hasta encontrar la soluci\'on \'optima o determinar que el problema no tiene soluci\'on. Parte de un v\'ertice de la regi\'on factible y "salta" a v\'ertices adyacentes que mejoren lo encontrado hasta encontrar la condici\'on de salida.

\subsection{Complejidad}
En el peor de los casos el S\'implex podr\'ia probar tantas bases como el m\'etodo exhaustivo. (Solo en ejemplos montados para que falle.) Normalmente va a encontrar la soluci\'on \'optima en $3n$ intentos.

\subsection{Propiedades Fundamentales}
\begin{itemize}
\item \textbf{Convergencia:} El algoritmo converge a la soluci\'on \'optima en un n\'umero finito de pasos (en la mayor\'ia de los casos pr\'acticos)
\item \textbf{Complejidad:} En el peor caso tiene complejidad exponencial, pero en la pr\'actica es muy eficiente
\item \textbf{Optimalidad:} Garantiza encontrar la soluci\'on \'optima global para problemas convexos
\item \textbf{Factibilidad:} Mantiene la factibilidad en cada iteraci\'on
\end{itemize}

\subsection{Aplicaciones}
S\'implex se usa en aplicaciones profesionales como la optimizaci\'on de recursos empresariales, la log\'istica y la gesti\'on de proyectos, donde se usa para encontrar la soluci\'on m\'as eficiente a problemas complejos con m\'ultiples variables y restricciones.

\subsection{Descripci\'on del M\'etodo}
El m\'etodo Simplex opera movi\'endose entre v\'ertices adyacentes del poliedro factible, mejorando el valor de la funci\'on objetivo en cada paso hasta alcanzar el \'optimo.

\section{Pasos del M\'etodo Simplex}
\begin{enumerate}
\item \textbf{Formulaci\'on del problema:}
\begin{itemize}
\item Define la funci\'on objetivo a maximizar o minimizar y las restricciones lineales asociadas.
\item Aseg\'urate de que todas las variables de decisi\'on sean no negativas.
\end{itemize}

\item \textbf{Conversi\'on a forma est\'andar:}
\begin{itemize}
\item Transforma todas las restricciones de desigualdad en igualdades introduciendo variables de holgura (sumar una variable para restricciones $\leq$) o variables de exceso (restar una variable para restricciones $\geq$).
\item Aseg\'urate de que la funci\'on objetivo tambi\'en se iguale a cero.
\end{itemize}

\item \textbf{Creaci\'on de la tabla inicial:}
\begin{itemize}
\item Organiza el problema en una tabla llamada Tabla Simplex.
\item La tabla inicial se construye con las variables del problema y una matriz identidad formada por las variables de holgura/exceso.
\end{itemize}

\item \textbf{Iteraci\'on:}
\begin{itemize}
\item \textbf{Selecci\'on de la variable entrante:} Elige la columna con el valor m\'as negativo en la fila de la funci\'on objetivo (para maximizar) o el m\'as positivo (para minimizar). Esta es la columna que entra en la base.
\item \textbf{Selecci\'on de la variable saliente:} Calcula la raz\'on entre el lado derecho de cada restricci\'on y el valor correspondiente en la columna de la variable entrante. La fila con el cociente m\'as peque\~no es la que sale de la base.
\item \textbf{Pivoteo:} Realiza operaciones de escalerizaci\'on para hacer que el elemento en la intersecci\'on de la fila entrante y la columna saliente sea $1$ y todos los dem\'as elementos en esa columna sean $0$. Esto actualiza la tabla y la soluci\'on.
\end{itemize}

\item \textbf{Comprobaci\'on de la soluci\'on \'optima:}
\begin{itemize}
\item Verifica si la fila de la funci\'on objetivo en la tabla ya no contiene coeficientes negativos (para maximizar) o positivos (para minimizar).
\item Si es as\'i, la soluci\'on es \'optima y se puede leer directamente de la tabla.
\item Si no, repite el proceso de iteraci\'on.
\end{itemize}
\end{enumerate}
\section{Problema Original}

\subsection{Formulación Matemática}
\textbf{Problema de Maximización}

\[ \text{Maximizar } Z = 0x_{1} + 2x_{2} \]

\textbf{Sujeto a:}
\begin{align*}
1x_{1} -1x_{2} &\leq 4 \\
-1x_{1} + 1x_{2} &\leq 1
\end{align*}

\textbf{Con:}
\[ x_{1} \geq 0, \quad x_{2} \geq 0 \]

\section{Tabla Final}

\textbf{Error:} No se pudo generar la tabla final.

\section{Solución Óptima}

\textbf{PROBLEMA NO ACOTADO}

\subsection{Explicación del Problema No Acotado}

Un problema de programación lineal se considera \textbf{no acotado} cuando la función objetivo puede mejorar indefinidamente sin violar ninguna restricción.

\textbf{Condiciones para no acotamiento:}
\begin{itemize}
\item Existe al menos una variable que puede aumentar indefinidamente
\item Todos los coeficientes en la columna pivote son negativos o cero
\item No hay restricciones que limiten el crecimiento de la variable
\end{itemize}

\textbf{Interpretación práctica:}
\begin{itemize}
\item En problemas de maximización: La ganancia puede ser infinita
\item En problemas de minimización: El costo puede disminuir indefinidamente
\item Suele indicar un error en la formulación del problema
\item Puede significar que faltan restricciones importantes
\end{itemize}

\textbf{Solución:} Revisar la formulación del problema y verificar que todas las restricciones necesarias estén incluidas.

\end{document}
