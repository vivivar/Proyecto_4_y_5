\documentclass[12pt]{article}
\usepackage[utf8]{inputenc}
\usepackage[spanish]{babel}
\usepackage{amsmath,amssymb}
\usepackage{booktabs}
\usepackage{xcolor}
\usepackage[table]{xcolor}
\usepackage{graphicx}
\usepackage{geometry}
\usepackage{colortbl}
\usepackage{array}
\usepackage{tikz}
\usetikzlibrary{shapes,arrows,positioning,calc}
\geometry{margin=2.5cm}
\usepackage{fancyhdr}
\usepackage{multirow}
\usepackage{float}
\setlength{\headheight}{14.5pt}
\pagestyle{fancy}
\fancyhf{}
\rhead{Investigación de Operaciones}
\lhead{Método Simplex}

\definecolor{basecolor}{RGB}{200,255,200}
\definecolor{entracolor}{RGB}{255,150,150}
\definecolor{salecolor}{RGB}{255,200,100}
\definecolor{pivotecolor}{RGB}{200,100,255}
\definecolor{empatecolor}{RGB}{255,255,100}
\definecolor{calculocolor}{RGB}{100,255,255}

\title{Resultados del Método Simplex\\
\large Problema: \textbf{NoFactible}}
\author{
Emily Sánchez\\
Viviana Vargas\\
\\
Curso: Investigación de Operaciones\\
Semestre II: 2025
}
\date{\today}

\begin{document}

\maketitle
\thispagestyle{empty}

\vfill
\begin{center}
\textbf{George Dantzig (1914-2005)}
\\
Creador del Método Simplex
\end{center}
\vfill

\newpage
\tableofcontents
\newpage
\section{El Algoritmo Simplex}

\subsection{Historia}
El método Simplex fue desarrollado por George Dantzig en 1947 mientras trabajaba para la Fuerza Aérea de los Estados Unidos. \\

Es uno de los algoritmos más importantes en la historia de la optimización matemática y ha sido fundamental en el desarrollo de la programación lineal. Usa operaciones sobre matrices hasta encontrar la solución óptima o determinar que el problema no tiene solución. Parte de un vértice de la región factible y "salta" a vértices adyacentes que mejoren lo encontrado hasta encontrar la condición de salida.

\subsection{Método de la Gran M}
El método de la Gran M se utiliza cuando el problema tiene restricciones de tipo $\geq$ o $=$ que requieren variables artificiales. Se asigna un coeficiente $M$ muy grande en la función objetivo para las variables artificiales, donde:
\begin{itemize}
\item Para \textbf{maximización}: $M$ es negativo grande ($-M$)
\item Para \textbf{minimización}: $M$ es positivo grande ($+M$)
\item El valor de $M$ utilizado es: $1000000$
\end{itemize}
Esto fuerza a las variables artificiales a salir de la base en la solución óptima.

\subsection{Propiedades Fundamentales}
\begin{itemize}
\item \textbf{Convergencia:} El algoritmo converge a la solución óptima en un número finito de pasos
\item \textbf{Optimalidad:} Garantiza encontrar la solución óptima global para problemas convexos
\item \textbf{Factibilidad:} Mantiene la factibilidad en cada iteración
\end{itemize}

\section{Formulación del Problema}

\textbf{Problema:} NoFactible\\
\textbf{Tipo:} Maximización\\
\textbf{Número de variables:} 2\\
\textbf{Número de restricciones:} 4

\subsection{Función Objetivo}
\[
Maximizar Z = 3x_{1} + 2x_{2}
\]

\subsection{Restricciones}
\[
\begin{cases}
0.0250x_{1} + 0.0166x_{2} \leq 1 \\
0.0200x_{1} + 0.0200x_{2} \leq 1 \\
x_{1} \geq 30 \\
x_{2} \geq 20
\end{cases}
\]

\subsection{Restricciones de No Negatividad}
\[
x_{1} \geq 0, \quad x_{2} \geq 0
\]

\section{Método de Solución}

Se utilizó el \textbf{método de la Gran M} debido a la presencia de restricciones de tipo $\geq$ o $=$.

\begin{itemize}
\item Valor de M utilizado: $\mathbf{1000000}$
\item Se introdujeron variables artificiales para las restricciones relevantes
\item El método garantiza encontrar una solución factible si existe
\end{itemize}

\subsection{Tabla Inicial del Método Simplex}

\textbf{Variables básicas:} $s_{1}$, $s_{2}$, $a_{1}$, $a_{2}$\\

\begin{center}
\small
\begin{tabular}{|c|c|c|c|c|c|c|c|c|c|}
\hline
\textbf{Base} & \textbf{$x_{1}$} & \textbf{$x_{2}$} & \textbf{$s_{1}$} & \textbf{$s_{2}$} & \textbf{$e_{1}$} & \textbf{$e_{2}$} & \textbf{$a_{1}$} & \textbf{$a_{2}$} & \textbf{b} \\
\hline
Z & -1,0M & -1,0M & 0 & 0 & M & M & -2M & -2M & -50M \\
\hline
$s_{1}$ & 0.0250 & 0.0166 & 1 & 0 & 0 & 0 & 0 & 0 & 1 \\
\hline
$s_{2}$ & 0.0200 & 0.0200 & 0 & 1 & 0 & 0 & 0 & 0 & 1 \\
\hline
$a_{1}$ & 1 & 0 & 0 & 0 & -1 & 0 & 1 & 0 & 30 \\
\hline
$a_{2}$ & 0 & 1 & 0 & 0 & 0 & -1 & 0 & 1 & 20 \\
\hline
\end{tabular}
\end{center}

\textbf{Nota:} Se utilizó el método de la Gran M con $M = 1000000$\\
\textbf{Variables artificiales:} $a_{1}$, $a_{2}$\\

\section{Iteraciones del Método Simplex}

\subsection{Tabla Intermedia}

\textbf{Iteración:} 1\\
\textbf{Variables básicas:} $s_{1}$, $s_{2}$, $a_{1}$, $a_{2}$\\
\textbf{Variable que entra:} $?$\\
\textbf{Cálculo de razones para seleccionar pivote:}
\begin{itemize}
\item Fila 1: $\frac{1}{0.0250} = 40$
\item Fila 2: $\frac{1}{0.0200} = 50$
\item Fila 3: $\frac{30}{1} = 30$
\item \textbf{Razón mínima:} 30 (Fila 3)
\end{itemize}


\begin{center}
\small
\begin{tabular}{|c|c|c|c|c|c|c|c|c|c|}
\hline
\textbf{Base} & \cellcolor{entracolor}\textbf{$x_{1}$} & \textbf{$x_{2}$} & \textbf{$s_{1}$} & \textbf{$s_{2}$} & \textbf{$e_{1}$} & \textbf{$e_{2}$} & \textbf{$a_{1}$} & \textbf{$a_{2}$} & \textbf{b} \\
\hline
Z & \cellcolor{pivotecolor}-1,0M & -1,0M & 0 & 0 & M & M & -2M & -2M & -50M \\
\hline
$s_{1}$ & \cellcolor{entracolor}0.0250 & 0.0166 & 1 & 0 & 0 & 0 & 0 & 0 & 1 \\
\hline
$s_{2}$ & \cellcolor{entracolor}0.0200 & 0.0200 & 0 & 1 & 0 & 0 & 0 & 0 & 1 \\
\hline
$a_{1}$ & \cellcolor{entracolor}1 & 0 & 0 & 0 & -1 & 0 & 1 & 0 & 30 \\
\hline
$a_{2}$ & \cellcolor{entracolor}0 & 1 & 0 & 0 & 0 & -1 & 0 & 1 & 20 \\
\hline
\end{tabular}
\end{center}

\vspace{0.2cm}
\textbf{Significado de colores:}
\begin{itemize}\small
\item \textcolor{entracolor}{\blacksquare} Variable que entra
\item \textcolor{salecolor}{\blacksquare} Variable que sale
\item \textcolor{pivotecolor}{\blacksquare} Elemento pivote
\end{itemize}
\subsection{Tabla Intermedia}

\textbf{Iteración:} 2\\
\textbf{Variables básicas:} $s_{1}$, $s_{2}$, $a_{1}$, $a_{2}$\\
\textbf{Variable que entra:} $x_{1}$\\
\textbf{Cálculo de razones para seleccionar pivote:}
\begin{itemize}
\item Fila 1: $\frac{1}{0.0250} = 40$
\item Fila 2: $\frac{1}{0.0200} = 50$
\item Fila 3: $\frac{30}{1} = 30$
\item \textbf{Razón mínima:} 30 (Fila 3)
\end{itemize}

\textbf{Operación de pivote:}
\begin{itemize}
\item \textcolor{entracolor}{\textbf{Variable que entra:}} $x_{1}$
\item \textcolor{salecolor}{\textbf{Variable que sale:}} $a_{1}$
\item \textcolor{pivotecolor}{\textbf{Elemento pivote:}} $1$
\item \textbf{Posición:} Fila 3, Columna 1
\end{itemize}

\begin{center}
\small
\begin{tabular}{|c|c|c|c|c|c|c|c|c|c|}
\hline
\textbf{Base} & \cellcolor{entracolor}\textbf{$x_{1}$} & \textbf{$x_{2}$} & \textbf{$s_{1}$} & \textbf{$s_{2}$} & \textbf{$e_{1}$} & \textbf{$e_{2}$} & \textbf{$a_{1}$} & \textbf{$a_{2}$} & \textbf{b} \\
\hline
Z & \cellcolor{entracolor}-1,0M & -1,0M & 0 & 0 & M & M & -2M & -2M & -50M \\
\hline
$s_{1}$ & \cellcolor{entracolor}0.0250 & 0.0166 & 1 & 0 & 0 & 0 & 0 & 0 & 1 \\
\hline
$s_{2}$ & \cellcolor{entracolor}0.0200 & 0.0200 & 0 & 1 & 0 & 0 & 0 & 0 & 1 \\
\hline
\cellcolor{salecolor}$a_{1}$ & \cellcolor{pivotecolor}1 & \cellcolor{salecolor}0 & \cellcolor{salecolor}0 & \cellcolor{salecolor}0 & \cellcolor{salecolor}-1 & \cellcolor{salecolor}0 & \cellcolor{salecolor}1 & \cellcolor{salecolor}0 & \cellcolor{salecolor}30 \\
\hline
$a_{2}$ & \cellcolor{entracolor}0 & 1 & 0 & 0 & 0 & -1 & 0 & 1 & 20 \\
\hline
\end{tabular}
\end{center}

\vspace{0.2cm}
\textbf{Significado de colores:}
\begin{itemize}\small
\item \textcolor{entracolor}{\blacksquare} Variable que entra
\item \textcolor{salecolor}{\blacksquare} Variable que sale
\item \textcolor{pivotecolor}{\blacksquare} Elemento pivote
\end{itemize}
\subsection{Tabla Final - Solución Óptima}

\textbf{Variables básicas:} $x_{2}$, $s_{2}$, $x_{1}$, $a_{2}$\\

\begin{center}
\small
\begin{tabular}{|c|c|c|c|c|c|c|c|c|c|}
\hline
\textbf{Base} & \textbf{$x_{1}$} & \textbf{$x_{2}$} & \textbf{$s_{1}$} & \textbf{$s_{2}$} & \textbf{$e_{1}$} & \textbf{$e_{2}$} & \textbf{$a_{1}$} & \textbf{$a_{2}$} & \textbf{b} \\
\hline
Z & 0 & 0 & 60,2M & 0 & 1,5M & M & -2,5M & -2M & -4,9M \\
\hline
$x_{2}$ & 0 & 1 & 60.2410 & 0 & 1.5060 & 0 & -1.5060 & 0 & 15.0602 \\
\hline
$s_{2}$ & 0 & 0 & -1.2048 & 1 & -0.0101 & 0 & 0.0101 & 0 & 0.0988 \\
\hline
$x_{1}$ & 1 & 0 & 0 & 0 & -1 & 0 & 1 & 0 & 30 \\
\hline
$a_{2}$ & 0 & 0 & -60.2410 & 0 & -1.5060 & -1 & 1.5060 & 1 & 4.9398 \\
\hline
\end{tabular}
\end{center}

\textbf{Solución:}
\begin{align*}
x_{1} &= 30 \\
x_{2} &= 15.0602
\end{align*}

\textbf{Problema No Factible:} No existe solución que satisfaga todas las restricciones.\\

\section{Resultados}

\subsection{Solución Encontrada}

\subsection{Explicación del Problema No Factible}

Un problema de programación lineal se considera \textbf{no factible} cuando no existe ninguna solución que satisfaga todas las restricciones simultáneamente.

\textbf{Causas comunes:}
\begin{itemize}
\item Restricciones contradictorias
\item Región factible vacía
\item Variables artificiales permanecen en la base con valor positivo
\end{itemize}

\section{Conclusión}

El problema es \textbf{no factible}.\\
No existe ninguna solución que satisfaga todas las restricciones simultáneamente.\\
\textbf{Mensaje:} El problema no tiene solución factible (variables artificiales en la base con valor positivo)\\

\textbf{Recomendaciones:}
\begin{itemize}
\item Revise las restricciones del problema.
\item Puede haber conflictos entre las restricciones.
\end{itemize}

\end{document}
